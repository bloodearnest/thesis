\chapter{Networks}
\label{CHAPTER:NETWORKS}
\centerline{\rule{149mm}{.02in}}
\vspace{2cm}


%%%%%%%%%%%%%%%%%%%%%%%%%%%%%%%%%%%%%%%%%%%%%%%%%%%%%%%%%%%%%%%%%%%%%%%

\section{Introduction and Overview}

\subsection{A Marketplace For The Grid}

A marketplace is traditionally a location at which traders meet to trade. Many
factors have contributed over time to the formation of marketplaces. For
example, social factors like language and culture, or geographical factors such
as presence or absence of natural resources and ease of travel and
transportation, can affect the trading relationships in a market.  Technology
has played a pivotal role in this area, and the 20th century saw
sweeping changes in the way in which marketplaces occur. The advent of the
internet has impacted markets hugely, as now there are few geographical limits
on potential trading partners and opportunities.

Whereas before markets needed to have central locations to facilitate
communication between traders, modern communication technology allows for
potentially anybody to trade with anyone else.  This has the effect of
decentralising marketplaces in terms of geography to some degree. Centralised
marketplaces are still useful for communication and trading, but they are now
no longer as necessary as they once were. This suggest the possibility of
building a decentralised market place on the internet for use with grid
systems.

One aspect of marketplaces that has not changed is the fact that trading is
still a human activity (for the moment). This implies trading relationships
that are based on underlying social interactions between people. The network of
interactions between human traders is therefore a social network.

In this chapter, we examine various networks including social networks in order
to develop a suitable virtual marketplace for allocation on grid systems. We
look at recent development in network theory, particularly statistical analysis
of networks. We also look at different types of network, particularly social
networks, which form the basis of real world market places. We are particularly
interested in whether or not the properties of social networks make for more
efficient marketplaces. For reference see \cite{net-newman03-networks} for a
full review of network theory, from which much of this chapter has been
distilled.

\section{Network Characteristics}

A network is simply a set of nodes (vertices) connected by some number of links
(edges). The London tube map is an excellent example of a network, with each
station being a node and each section of tube line being an link. The possible
connection topologies for even a small number of nodes is very large.

\subsection{Statistical Analysis}

Large networks are complex to analyse. Beyond a few hundred nodes, simple
visualisation techniques are generally unhelpful for analysis. Network theory
uses statistical analysis of large networks in order to properly understand
their structure and behaviour. Some of the key properties are defined below.

\begin{itemize}
  
  \item Degree: the number or links a node has. A network's mean degree and
    degree distribution can be key indicators of the properties of that
    network. 
    
  \item Degree Correlation: the correlation of linked nodes' degrees can be a
    useful measure also. A high degree correlation indicates a well connected
    node is connected to other well connected nodes, and a low degree
    correlation indicates the opposite, that a highly connected node is
    connected to badly connected nodes.

  \item Path Length: number of links on the shortest path between any two
    nodes. Of particular interest is the average path length for a network,
    which is an indication of how fast information can spread between nodes.
  
  \item Transitivity: also called clustering, it represents the probability a
    node's linked nodes are also linked. Measured as a clustering co-efficient
    $C$ which is the mean proportion of maximum possible linked neighbour nodes
    clustering coefficient measures the probability that your friends will be
    friends with each other on a given network.

\end{itemize}


%%%%%%%%%%%%%%%%%%%%%%%%%%%%%%%%%%%%%%%%%%%%%%%%%%%%%%%%%%%%%%%%%%%%%%%


\section{Types of Network}


\begin{itemize}
  
  \item Random Networks.\\
    
    The classic basic random network is Erd{\"{o}}s and
    R{\'{e}}nyi's\cite{net-erdos59-random} random graph. This is defined by the
    probability $p$ that any two nodes are connected. These networks typically
    have a small average path length and low transitivity. These types of
    networks are useful as a base with which to compare other networks.

  \item Regular Networks.\\

    A regular network has a uniform structure of links for each node. For
    example, a two dimensional lattice where each node has a link to its north,
    south, east, and west neighbours. Another common regular networks is a ring,
    where the nodes are placed in a ring and each node is connected to the
    nodes on either side of it in the ring. The are simple to visualise and
    understand. 

  \item Small World Networks.\\
    
    First formalised in \cite{net-watts98-smallworld},
    small world networks are based on a regularly connected network, such as a
    grid or ring, and randomly rewiring some proportion of links. This
    maintains the high transitivity of the original regular network, while
    significantly lowering the average path length.  Named after the ``small
    world`` effect, the idea any two people can be linked one to the other
    via a short number of other people (network hops).
    
  \item Scale Free Networks.\\

    In a regular lattice network, for example, each node has a degree of 4, and
    the degree distribution is therefore uniform.  However, in many real world
    networks, the degree distribution is not uniform. Barab{\'{a}}si and Albert
    \cite{net-barabasi99-scaling} showed that networks with an exponential
    degree distribution exhibited similar characteristics as small world
    networks, such as high transitivity and low average path length, with the
    additional property of negative degree correlation.  They proposed a
    preferential attachment algorithm for creating such networks, in which a
    new node joining the network is more likely to connect to a well connected
    network than a poorly connected one.  In a sense, scale-free networks are
    in part hierarchical, with high degree nodes at the top of the
    hierarchy, and low degree nodes at the bottom. The internet and the World
    Wide Web are good examples of scale-free networks, and show the scalability
    advantages of such topologies. 

  \item Social Networks.\\
    
    Social networks have traditionally been hard to quantify. In recent years,
    examples of networks such as high school student friendships and scientific publication
    collaborations have been available to study and have allowed interesting
    insights. Newman and Park\cite{net-newman03-social} show that social
    networks have a particularly unique combination of properties, high
    transitivity (your friends are also friends) and high degree correlation
    (popular people know other popular people, loners know other loners). This
    type of network is particularly relevant to this study as economic networks
    are base on social networks. However, a social network may not be the best
    topology for an online market - other topologies may be more efficient,
    but previously have not been feasible to implement.

\end{itemize}


%%%%%%%%%%%%%%%%%%%%%%%%%%%%%%%%%%%%%%%%%%%%%%%%%%%%%%%%%%%%%%%%%%%%%%%


\section{A Grid Marketplace}

An online grid market place requires an underlying network of trader
relationships to function. Random, small world, scale-free and social network
topologies all have properties that maybe useful in implementing a market.  A
previous study \cite{net-noble04-information} to this investigated the effects
of topology on a simple economic market, using small world and scale free
networks. BitTorrent \cite{net-cohen03-bittorrent} utilises random networks to
great effect in achieving scalable bandwidth distribution. It was found that
small world and scale free networks had regions in the space of their construction
parameters that provided a better transmission of information, which is a
starting point for our investigations. Social networks have not been deeply
investigated when it comes to computational economics, so that is a
potential avenue to pursue.

In addition to the information transmission aspect of a topology, our focus on
a real world problem like resource allocation on a grid brings in performance
questions. How high a degree can an individual node support before the
communication cost become two large? Can the topology used increase affect
value? These are additional factors to explore in our investigation.
