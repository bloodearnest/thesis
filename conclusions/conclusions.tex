\chapter{Conclusions and Further Work}
\label{CHAPTER:CONCLUSION}
\centerline{\rule{149mm}{.02in}}
\vspace{2cm}


\section{Conclusions}

\subsection{Objectives}

In this thesis, we have achieved the following objectives, as outlined in
Chapter \ref{CHAPTER:INTRO}.

\begin{itemize}

  \item We provide a detailed description of economic mechanism, and have
    examined their suitability for use in a grid environment.
    
  \item We have proposed a fully decentralised economic allocation mechanism for
    grid systems.

  \item We have implemented a suitable simulation tool and uses it 
    for examining our proposed mechanism.

  \item We implemented and analysed functionality and performance of the proposed
    allocation mechanisms in simulation.

\end{itemize}

Whilst the basic mechanism proposed here has some drawbacks, it is a first step
towards producing such systems. There is much that could be done to improve the
system, which is outline in section \ref{SEC:CONC:FW}

\section{Thesis Contribution}

We propose a novel fully decentralised system for resource allocation on grid
systems, using an economic metaphor. We have shown that this is a feasible
option for such a scheme, and in key aspects such as scalability, it has
advantages over the current centralised approach. While there are many areas of
improvement the may be needed to make such a system a suitable solution, we
have carried out initial exploratory work and laid a foundation for future
development.

We also explored the application of basic network theory to the concept of
trading networks, which has not been extensively investigated, particularly at
large network sizes.

\section{Further Work}
\label{SEC:CONC:FW}

\subsection{Other Allocation Mechanisms}

To further understand the degree which the quality of the allocation maybe
sub-optimal, a comparison would need to be made to the theoretically optimal
solution. This would require extending the simulation to calculate the optimal
allocation, ideally side by side with the economic system. Additionally, it would
also be of value to compare with the equivalent centralised approach for this
grid model, both the allocation and system performance. This would require the
implementation of an aggregated central grid registry similar to existing grid
systems.

\subsection{Grid Model}

The grid model could refined  to include aspects we have not discussed her. An
obviously useful addition would be the inclusion of a system for estimating job
execution times, with some mechanism for error. This estimated completion time
could be added into the trading mechanism as an additional price metric, with a
deadline being the effective limit price. This would allow buyers to prioritise
on price of time depending on their needs.

Additionally, the notion of a fixed Job size could be relaxed, and Jobs could
be allowed to run on less resource capacity but take longer to complete, or on
more capacity for a faster completion time. As not all applications are easily
altered in this manner, a scalability factor could be introduce to represent
this. A scalability of one could indicate that a job is trivially parallel and
can easily run on more or less resource for a linear adjustment in execution
time. A scalability of zero would mean that a job cannot be run on more or less
resource at all, with value in between modifying the estimated execution accordingly

\subsection{Economic Market}

The sealed bid auction is simple to implement, which is the reason is has been
used in this study. However, the CDA has some very attractive properties, and
could potentially out perform the sealed bid auction, although it would also
require a greater degree of communication. Additional potential auction
mechanism that may be of value to investigate are Dutch and retail auctions.

While it is interesting that ZI traders can still perform well in this system,
reinforcing the original concept, they are clearly sub-optimal. A simple
expansion would be to include ZIP and GD based mechanisms for estimating market
price. Given the limitations of ZI with respect to non-standard market
conditions, this would be an important addition if different market conditions
are used.

The markets so far have been static, with fixed demand and supply and a static
seller base. Real grid systems would need to adapt to changes in demand and
supply, and with sellers and their resources appearing and disappearing. The
ability of the market to recover after a shock would be an important
requirement for a real grid system. This would also highlight the differences
between the different auctions mechanisms.


\subsection{Network Topologies} 

The network in our study is fixed statically at the start of the simulation. In
a real grid environment, nodes could be leaving and joining the network at any
point. This would require the development of algorithms to attach and detach
nodes from a network, whilst still maintain the topological structure desired.

The calculation of the network statistics describe in chapter
\ref{CHAPTER:NETWORKS} would be a useful addition, potentially allowing further
insight into the effects of the various topologies.

