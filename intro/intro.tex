\chapter{Introduction}
\label{CHAPTER:INTRO}
\centerline{\rule{149mm}{.02in}}
\vspace{2cm}


%%%%%%%%%%%%%%%%%%%%%%%%%%%%%%%%%%%%%%%%%%%%%%%%%%%%%%%%%%%%%%%%%%%%

\section{Motivation}

\subsection{Computational Grids}

The area of ``grid computing'' is a recent development from the last 20 years
of research into computing resource provision.  It denotes a concept of
computers as a utility, where computational resource can be provided as-and-when
needed transparently to large numbers of people. It is similar in concept to
other utilities, such as water or electricity, indeed the grid name owes in
part to its similarity in concept to the National Grid. A grid system would
allows users to utilise massive compute resources in as easily and as
transparently as we plug something into a socket at home and turn it on.

While allowing different people to use computing resources is nothing new, the
grid concept takes the concept to a new level, in several ways. Firstly, the
sheer scale is much larger. Previously, a few hundred machines, or at the most
a few thousand, have been harnessed together in some sort of grid-like system.
The grid concept takes these numbers in to the tens of of thousand, or even
millions. Secondly, and more importantly, it relies on the sharing of multiple
resources owned and administered by different organisations or individuals.  In
previous resource sharing systems, the resources are all owned by a single
entity. Grid computing, however, aims to allow many resource-owners to
participate in sharing their resources together in a single unified manner.
This is the ultimate aim of grid computing, and presents some majors
challenges for researchers of grid computing. Arguably the most significant of
these, and the one with the fewest practical solutions, is resource allocation.

In a grid environment, a resource is some computational facility which users
wish to utilise for their applications and programs. The challenges of deciding
what users run which applications on whose resources are significant,
especially in the face of huge numbers of resources and different
administrative domains. Additionally, to achieve the full potential of
grid computing, resource allocation needs to be done in a fast, automatic,
robust and scalable manner.  

The current dominant approach is to have a central point that maintains the
state of all resources on the grid, and provides ways of selecting suitable
resources to use. This is the historical approach to allocating computing
resources, as most computational systems are centralised in nature. However,
this method is not particularly automatic, scalable or robust, at least in
current implementations, especially when you start to scale to thousands of
resources. Current resource allocations systems do not deliver the necessary
resource allocation capabilities to fulfil the aims the grid concept.


\subsection{Economic Resource Allocation}

The problem of allocating scarce resources is one that has ever been present in
history. A resource is scarce if there is not enough supply of that resource to
meet the corresponding demand for it.  The raw method used by nature for
allocating such resources is though competition and survival.  Humans have
developed more sophisticated systems for facilitating this allocation, namely
economics.

Economics provides a system for allocating scarce resources between different
parties using understood mechanisms to swap (trade) one resource for another,
usually money in exchange for some other resource.  Many varying and complex
economic systems have arisen of the centuries of human history to facilitate
the exchange of scarce goods, and in recent centuries, these system have been
much examined in their function and performance. The particular advantages of
economic systems are that they are adaptive to change, and highly scalable and
robust.  However, they do not guarantee an optimal allocation.

They property of economic systems that gives rise to these factors is that they
are essentially decentralised. Whilst many existing economic systems utilise a
central mechanism for efficiency or simplicity, base economic systems developed
from individuals acting independent of any central imperative. This
decentralisation is present in two distinct ways. Firstly, individual
participants in an economic system, that is, the traders, are self-interested.
The system simply provides them with an incentive to trade. There is no
mandated co-operation, other than conforming to the common trading protocol,
each individual agent is its own master, and utilises its own resources to
participate in the trading.  Secondly, no single trader has a complete picture
of the whole system.  Traders have knowledge of their own needs or resources
which is known only to themselves. Additionally, traders usually only
participate in a subset of all the trade interactions occurring in the system,
and thus have only a limited knowledge of other's trades.  Yet despite these
limitations, the interactions of many traders give rise to a global system
behaviour that has useful properties, especially when applied to the grid
environment.

\subsection{An Economic Grid}

There is a clear parallel between the needs of a resource allocation mechanism
for grid systems and the economic resource allocation paradigm. An economic
system for allocating resources on a grid could provide the scalability and
robustness required to achieve the grid concept, and overcome the limits of
current approaches.  It it also well suited to the problem of multiple
autonomous domains, as each resource can act as a self-interested agent in an
intuitive manner. It could provide a natural revenue model for for grid
systems, and area which is still largely speculation.

In order to achieve the potential scalability and robustness benefits of and
economic system, it needs to be as decentralised as possible. Such a system
would very different from the current approaches, representing an orthogonal
approach to grid resource allocation. If such a system could be implemented, it
could provide the necessary transparency and scalability needed to implement a
working grid system that embodies the aims of the grid concept.


%%%%%%%%%%%%%%%%%%%%%%%%%%%%%%%%%%%%%%%%%%%%%%%%%%%%%%%%%%%%%%%%%%%%%

\section{Objectives}

This thesis explores the potential for a fully decentralised economic resource
allocation scheme for a computational grid. The key objectives of this thesis
are as follows.

\begin{itemize}

  \item To provide a detailed description of economic mechanism, and examine
    their suitability for use in a grid environment.
    
  \item To propose a fully decentralised economic allocation mechanism for
    grid systems.

  \item To implement a simulation tool with a suitable a model of a grid system
    for examining the proposed mechanism.

  \item To implement the proposed allocation mechanisms in simulation, and
    analyse for functionality and performance, and suggest areas for future
    investigation.

\end{itemize}


\section{Thesis Contribution}

Whilst economic allocation schemes for computational resources and grid systems
have been developed previously, they are all centralised in nature, thus losing
some of the potential benefits of using such systems. The main contribution of
this work is explore the potential of a fully decentralised system, which is a novel
approach, especially in the grid field. We show that such an approach is not
only feasible, but has various advantages over the current approach.

In aiming to keep the system as simple as feasible, we also add to the
discussion on the relationship between trader rationale and market structure.
See section \ref{SEC:ECON:ZI} for more details.

We also apply network theory to our economic model. In most computational economic
systems, communication between traders is assumed to be either all-to-all or
random in nature. We look at different topologies for communication between
traders, and apply these findings to much larger networks than have previously
been studied.


%%%%%%%%%%%%%%%%%%%%%%%%%%%%%%%%%%%%%%%%%%%%%%%%%%%%%%%%%%%%%%%%%%%%%

\section{Thesis Overview}

In Chapter \ref{CHAPTER:GRID} we review the literature from the field of grid
computing, including the history and motivation of the grid concept, as well as
a critique of the current solutions to resource allocation.

In Chapter \ref{CHAPTER:ECON}, we review basic economic theory and examines
previous work in modelling economic systems computationally, as well as existing
work applying economics principles to computational resource management.

Chapter \ref{CHAPTER:NETWORKS} take a brief look at the field of statistical
network theory, including a variety of network types and properties.

Chapter \ref{CHAPTER:METHOD} outlines our proposed investigation, including a
basic model of a grid system, and presents a simple economic model for
market system and trader rational

In Chapter \ref{CHAPTER:RESULTS} we present the experiment details and initial
findings based on the proposed solution in the previous chapter.

In the final chapter, Chapter \ref{CHAPTER:CONCLUSION} concludes the study,
summarising the results findings and suggests areas for future work.


