\pagestyle{plain}
\pagenumbering{roman}
\setcounter{page}{1}

\begin{center}
    {\LARGE\bf Abstract}
\end{center}

Grid computing is the concept of harnessing the power of many computational
resources in a transparent manner. It is currently an active research area,
with significant challenges due to the scale and level of heterogenity
involved.  One of the key challenges in implementing grid systems is resource
alloaction.  Currently, centralised approaches are employed that have limited
scalability, which is a key factor in acheiving a usable grid system.

The field of economics is the study of allocating scarce resources using
economic mechanisms. Such systems can be highly scalable, robust and adaptive.
There is also a natural fit of the economic allocation metaphor to grid
systems, given the diversity of autonomy of grid resources.

We propose that an economic systems is a suitable mechanism for grid resource
allocation. We propose a simple market mechansim to explore this idea. Our
system is a fully decentralised economic allocation scheme, which aims to
acheive a high degree of scalability, and easily allows resources to retain
their autonomy.

We implement a simulation of a grid system to analyse this system, and explore
its perfomance and scalabilty. We use a network to facilitate communication
between participating agents, and we pay particular attention to the topology of
the network between participating agents, examining the effects of different
toplogies on the performance of the system.

\newpage

