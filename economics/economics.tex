\chapter{Economic Resource Allocation} 
\label{CHAPTER:ECON} 
\centerline{\rule{149mm}{.02in}}
\vspace{2cm}

\section{Introduction and Overview}

Economics is the study of the allocation of scarce resources between competing
alternate uses. The allocation decision making is distributed amongst the many
interested parties, whether individuals and organisations, with no overall
central control.  It is driven by the availability of a resource (supply) and
the need for that resource (demand). In particular, we are interested in
\emph{microeconomics}, the study of individual traders' behaviour and
interactions, and their aggregated effects. 


We start with a summary of simple economic principles.  There is much published
work within the field of economics, and we focus on specifically
on decentralised economic mechanisms. Most real-world and theoretical economic
studies are based on some centralised mechanism, and as such are not directly
relevant to our study. Also covered in this chapter is a look at specific
computational market-based control algorithms that have been developed in the
last 20 years.

\section{Modelling an Economic Market}

The term 'market' is used to describe the interactions of traders trading a
particular commodity. The are many different types of markets, selling many
different products, but the basic principles of a market are well defined. Here
we introduce some key concepts about economic markets, and go on to look at
three key components of a market; the traders who act in the market, the rules
of the market (auction mechanism) and a market place in which the interaction
occurs.  Much of this section has been distilled from
\cite{eco-nicholson00-microeconomics}

\subsection{Supply, Demand, and Market Equilibrium}
\label{SEC:ECON:EQUILIBRIUM}

The basic model of supply and demand is well understood. A larger supply of a
resource than demand for it will lead to the price falling as sellers of the
resource compete with each other by offering lower prices to make a sale.
Likewise, a high demand relative to supply would lead to the price rising as
buyers compete to meet their demand.

A key idea in microeconomics is that of market equilibrium. This is the price
in a market for which the supply and demand are equal, and all supply of a
resource is sold and all demand for that resource is met. Any other price would
result in a surplus or shortfall of resource. The aggregate price of a
transaction in a market will tend towards this point over time. The buyer does
not want to spend any more than necessary to meet their demand, and sellers do
not want to waste their own resources to produce resource that is not needed.
It is an \emph{efficient} market. The further the aggregate price is from the
equilibrium price, the more inefficient the market.

Note that there is not always a equilibrium. Factors governing the nature of
producing the resource, or other outside influences, can result in markets that
have constant shortfall or surplus of resource. 

A particular resource's supply and demand do not remain constant. As they
change, the market price will track the equilibrium price, as this is in the
interests of all traders individually. Thus a market will usually tend towards
this 'optimal' efficiency. 

The rate of change of demand and supply varies in different types of markets.
Some are slow and static with well understood supply and demand models, such as
traditional resources like stone and timber. Others are highly dynamic and volatile (e.g. stock, commodities). This
difference in dynamism leads to many different types of market. A less dynamic
market can use a longer process to agree on a price, for example, while a highly
dynamic market needs to utilise a faster method. See section
\ref{SEC:ECON:AUCTIONS} for more examples.

This concept of an 'optimal' price for a particular supply/demand value is key
in evaluating economic systems, particularly when a market is concerned with
global allocative efficiency, as is the case in Grid systems.


\subsection{Market Traders} 
\label{SEC:ECON:TRADERS}

Traders are the individual agents that participate in the market. A trader has
to decide on a public quote to give to other traders, representing the value at
which the trader is willing to trade. These quotes are communicated to other
traders, and if a trader see a quote that is within there private valuation,
they can accept the quote and exchange goods and money.

In real world markets, traders are people, who make decisions about trading
according to some internal rationale. It is this rationality which has been the
focus of much study and investigation. How does a trader decide what to bid?
What information is available to the trader to make the decision? How does the
trader adapt to changes in demand/supply?

Traders have the following basic characteristics;

\begin{itemize} 
  
  \item{\emph{Self-interested}}

    A trader is fundamentally selfish and interested in its own utility or
    profit.  While traders can cooperate, they do so only to their own
    advantage. No rational trader will knowingly trade at a loss

  \item{\emph{Autonomous}}

    A trader makes its own trading decisions, without the need for a central
    authority. Often some sort of central authority is utilised, but it is not
    essential.

  \item{\emph{Limited resources}} 

    A trader has basic limitations. They have a finite amount of resource to
    sell or money to purchase resource. If traders had very large resources, an
    economic system would be less necessary.

  \item{\emph{Limited knowledge}}

    The trader only has a limited view of the world: they are not omniscient.
    This view is formed from two main sources; public information that can be
    observed from the actions or communications of other traders, or is
    provided by a central arbiter, and private information, such as their own
    resources or valuation of a resource.

  \item{\emph{A rationale}}

    Traders have some way of using the information that they have to estimate
    the market price and formulate a price to quote at. This can be modelled
    mathematically for a mathematical model, or algorithmically for an agent
    based one, and can range from the simple (e.g. constant valuation) to
    highly complex (see \ref{SEC:ECON:TRADERS})

\end{itemize}

It is these characteristics that lend the trader metaphor to the grid
environment, as there are many similarities between traders and resources and
jobs, particularly in the self-interest of traders, jobs and resources.



\subsection{Auctions}
\label{SEC:ECON:AUCTIONS}

An auction is a set of well understood rules or conventions by which to
communicate quotes between traders and reach a price. Many forms have arisen
over time that have varying characteristics as needed by a particular situation
or market. The main recognised auction types, and the way in which they vary,
are summarised below. Much of this review is based on
\cite{eco-klemperer99-auction}.

\subsubsection{Classical Auctions}

There are four 'classical' auction types recognised in the literature, each
involving a single seller, and multiple buyers, or vice-versa.

\begin{itemize}

  \item \emph{English} An English auction is one of the most commonly
    recognised auction mechanisms.  It is a public auction with a single seller
    selling a single resource and multiple competing buyers. The buyers start
    quoting low and raise the quote until no other buyer is willing to quote
    higher. The buyer with the highest quote wins the auction.  Often a central
    independent 'auctioneer' is used to control the quote rises. The advantage
    of an English auction is that it gets the highest price for the seller. It
    is commonly used for very scarce and unique resources such as objects d'art
    and large houses. In other words, high demand and low supply.

  \item \emph{Dutch}\\
    A Dutch auction is similar to the English auction in that there is a single
    seller and multiple buyers, but the quoting is reversed. The seller starts high
    and then repeatedly lowers the quote until the first buyer accepts the price.
    It is very fast auction and is utilised in flower markets where speed is of the
    essence.

  \item \emph{Sealed-bid}\\
    In sealed-bid auctions the traders submit private
    quotes to a single opposing trader, who is a buyer with many sellers or
    vice versa. The best quote (highest or lowest, depending on whether buying
    or selling) wins the auction. Government contracts commonly use this form.
    A variations of this is to make the bidding iterative in rounds, whith the
    best bid of each round being communicated between rounds. 

  \item \emph{Vickery}\\
    The Vickery auction, or second price auction, is the same as a
    sealed-bid auction, where the best bidder wins the auction, but at the
    price of the second best bidder. This is semantically equivalent to an English
    auction, where the best price is not actually quoted, the winner wins when
    they offer just above the maximum price of the second best bidder.

\end{itemize}

\subsubsection{Other Auction Mechanisms}

Retail auctions, or posted price auctions, are the most common, being used in
most western consumer markets. The seller posts a price, and the buyer accepts or
not, take it or leave it.  The seller adjust the posted price occasionally.
This is used in high volume markets with low dynamics and is very scalable.
A local supermarket is a prime example.

Another auction mechanism that has been facilitated by computing power is the
call auction. In a call auction, all traders submit their trade prices to a
central independent ``order book''. The total market supply and demand are then
calculated, along with the equilibrium price, and all traders simultaneously
trade at this price. This is a fast and efficient market form that is common in
finance markets, but is completely centralised in nature, as the central order
system must be independent and trusted by all traders.

\subsubsection{Double Auctions}

A double auction is one where both buyers and sellers quote. The previous
markets are 'single' auctions in this sense, as only the buyer or seller
quotes, not both.  The most common form of this auction is a continuous double
auction. This is a many-to-many auction, commonly used in stock market trading
until the recently, when the advent of computing allowed more efficient
mechanisms to be used. Such auctions occur in a shared common space, where a
trader will shout out his quote, while simultaneously listening to quotes from
both other buyers and sellers. He then adjusts his quote according to what he
hears until his price meets that of an opposing trader and a trade is made.
This has the advantage of being efficient in the sense of fast adaptation to the
market equilibrium. It is used in highly dynamic markets such as stocks and
commodities. A common stipulation, often called the New York Stock Exchange
rule after its origin, requires that a quote persists after being given, and
subsequent quotes must improve on that quote.

\subsubsection{Summary}

Many different auctions exist, and can vary in different ways:

\begin{itemize}
  \item quotes can be private or public;
  \item quotes be communicated simultaneously or continuously;
  \item quotes can be given by the buyers, sellers or both;
  \item the number of quotes allowed can vary;
  \item the number of participants: one-to-one/one-to-many/many-to-many;
  \item the need for a central arbiter or not;
\end{itemize}

For our work, we will need auction mechanisms that can be decentralised, and
are suited to the Grid environment, as well as the usual desirable properties
such as efficiency.


\subsection{Terminology}

For the rest of this thesis, the following terms as defined as described below.

quote: a price offered from one trader to another for an amount of resource

bid: a price offered by a buyer to a seller

offer: a price offered by a seller to a buyer

transaction: a trade between two traders at an agreed on price

market price: the current aggregate price of a transaction.

equilibrium price: the theoretical price where supply and demand meet and the
market is most efficient


%%%%%%%%%%%%%%%%%%%%%%%%%%%%%%%%%%%%%%%%%%%%%%%%%%%%%%%%%%%%%%%%%%%%%%%

\section{Computational Economics}
\label{SEC:ECON:COMP_TRADERS}

The basic equilibrium model described is section \ref{SEC:ECON:EQUILIBRIUM}
above is tried and tested, and used frequently in the study of market behaviour
and dynamics. It lends itself to the mathematical modelling of markets, where
supply and demand functions can be specified and the model's equations solved
to provide results. This has been the dominant form of modelling markets in the
field of economics. However, many of these mathematical models have had to make
assumptions about the market to retain mathematical tractability.  For
instance, they may assume perfect rationality on the part of traders, or a
perfectly random communication between traders.  It often requires fixed
supply/demand curves, and thus it is difficult to individual model agents'
adaption to changing market conditions. The final state of the market can be
examined, but not the dynamic state as a market 'finds' equilibrium.  This is
very much a 'top-down' model, attempting the capture the entire behaviour of a
system in concrete manner.

The field of experimental economics represents a different approach to
understanding markets, and was established by Vernon Smith in
1962\cite{eco-smith62-competitive}. He created a simple experimental market,
and used real human traders to explore market dynamics. This approach allows
individual trader behaviour to be examined, which the mathematical models do
not.

With the emergence of cheap and accessible computation now available to
researchers, experimental computational modelling has become increasingly
popular in economics during the last 20 years. Whereas the size and complexity
of experiments in earlier work was tied to human limitations, copmutational
agents allow for many more options. Markets can actually be fully simulated and
traced, and the behaviours of individual traders defined and examined. This
allows us to look at the above assumptions in great detail, and study more
complex questions that were not feasible with previous models. They can provide
more detail about the rich dynamics that can be present in economic systems.  

This approach is a 'bottom-up' model, as the individual agents give rise to
system-level behaviour. It allows us to observe details like robustness and
individual traders actions, and explore areas that were previously infeasible to
examine.


\subsection{Modelling Human Traders}

Every year since 1999, the Trading Agent Competition\cite{eco-wellman99-tac}
has been held as the primary forum for research into trading agents.  The
competition's market is a complex multi-market simulation of a travel and
leisure economy, where traders compete to purchase flights, accommodation and
entertainment as holiday packages.  Successful agents have including
SouthamptonTAC\cite{eco-he05-tada}, and Walverine\cite{eco-cheng05-walverine}.

The paradigm for these agents it a 'top-down' approach to artificial
intelligence. That is, they attempt to capture human thought patterns at a high
level, and develop methods of implementing similar strategies computationally.
This means that the architecture of these agents is often sophisticated and
multi-layered, as human strategies are often complex and intricate.
Additionally, these agents are designed for a specific market that is well
defined and implemented.

Whilst it would be potentially feasible to implement similar algorithms in our
exploratory work, the agents are complex to implement and would require
refactoring to be more general. Therefore, we look to a different approach for
trader rationale than the TAC-orientated algorithms.

\subsection{Simple Trading Agents}
\label{SEC:ECON:ZI}

Gode and Sunder\cite{eco-gode93-zi} investigated the effect of rationality in
market performance. They formulated a Zero-Intelligence (ZI) trader model,
where agents bid randomly in a double auction market, with the constraint that
the traders' random bids must be within their limit prices (ZI Constrained, or
ZI-C). They found that even with no bidding strategy, markets with ZI-C traders
still reached equilibrium, and proposed that market structure was a partial
substitute for trader rationality. This model has been much investigated, and
been used as a baseline market predictor \cite{eco-farmer05-predictive} amongst
further investigations into general market efficiency
\cite{eco-gode97-efficiency}.

Cliff explored ZI traders further \cite{eco-cliff97-minimal},
\cite{eco-cliff97-zip} showing market situations in which ZI-C traders failed
to find equilibrium, such as markets with constant supply or demand curves.
This suggest that the accurate equilibriums reached by ZI-C traders were
artefacts of the original simulation supply and demand distributions. He
proposed an improved trading algorithm, called Zero Intelligence Plus, or ZIP
traders. ZIP traders use a simple error descent learning rule to adjust a
trader's perceived market price relative to their limit price. He found that
ZIP traders out performed ZI-C in finding market equilibrium, and were also
able to successfully find equilibrium in the market environments in which ZI-C
traders were unsuccessful. He went on to further optimise the ZIP algorithm
\cite{eco-cliff01-evolution} using evolutionary optimisation, including
developing a novel partially-double auction mechanism. 

In a similar development, the Gjerstad-Dickhaut (GD) trader model was
introduced in \cite{eco-gjerstad98-gd}, which defined a ``belief'' function for
a trader, representing a traders belief in the current market price.   This
function was calculated using a window of recent successful market
transactions, and used to select a price that would maximize utility. The GD
model has been extended by others, using dynamic programming to formulate the
belief function (GD eXtended or GDX)\cite{eco-tesauro02-gdx}, and by using
unsuccessful transactions as well as successful ones (Modified GD or MGD)
\cite{eco-tesauro01-mgd}. 

All the trading algorithms described above were developed using a small double
auction market with around 20 traders, and all-to-all connections between
traders, which whilst simple, which would not be feasible in a large scale.
Additionally, the bidding was conducted in distinct syncronised rounds and the
trading was split into trading days, after the real-world experimental
economics that it was based upon. However, Cliff found ZIP traders to work in
continuous trading environments \cite{eco-cliff01-days}, which would be
necessary for use in our grid environment.

%%%%%%%%%%%%%%%%%%%%%%%%%%%%%%%%%%%%%%%%%%%%%%%%%%%%%%%%%%%%%%%%%%%%%%

\section{Market Based Control}

Market Based Control is the field of Artificial Intelligence that looks at
using economic principles in system design, first proposed as a methodology by
Miller and Drexler under the moniker ``agoric open systems''
\cite{eco-miller88-markets} as a part of a collection of papers published in 1988
entitled ``The Ecology of Computation'', edited by Bernado Huberman
\cite{eco-huberman88-ecology}. Huberman also included some real world
implementations or economic systems, which are discusseb below along with other
related work. 

\subsection{Previous Work}

The Enterprise mechanism \cite{eco-malone88-enterprise} was an early job
allocation system that utilised a decentralised, first-price, seller quotes,
single-round, sealed-bid auction who's price metric was estimated job
completion time, with a cancellation protocol. This was optimised toward mean
job flow rate, or throughput, and compared favourably to comparable systems at
the time.  It utilised an all-to-all communication network, which scaled well
up to 90 machines. However, such a system would eventually hit a limit of
communication overheads for its all-to-all connections, especially if some
machines were located non-local networks with increased communication costs,
particularly for broadcasting communication. This limits its potential
scalability. Additionally, using a user supplied completion time as a price
metric encouraged users to be dishonest in their estimates in order to get
quicker scheduling, and while they addresses this with a penalty mechanism, it
would be a potential problem in applying their solution to the grid.

Drexler and Miller presented a similar scheme\cite{eco-drexler88-crm}. They
used a rental model to share a single machines processor and storage resources
using a sealed-bid second price auction through simulation. The include a
method of escalating bid prices over time to ensure a degree of fairness and to
provide opportunities to change a process's current bid.  The design was
specific to a single machine, dealing with garbage collection and pointer
management, but is an explorative study on the potential for economics systems,
but uses a much lower level model of computation that the grid.

Similar to to the Enterprise project, Waldspurger \textit{et al} developed
Spawn, a distributed computational economic allocation
system\cite{eco-waldspurger92-spawn}. This used multiple decentralised
single-round sealed-bid second-price auctions to allocate idle resource time.
It included a sophisticated sponsorship hierarchy for allocating a complex job
across multiple auctions. Whilst decentralised, it was not tested on large
numbers of machines and used an single domain all-to-all communication network
for trading, although it discusses the possibility of scaling using a random
network.

In 1996 Scott Clearwater reviewed several later MBC systems
\cite{eco-clearwater96-mbc}, that built upon this idea, including systems to
auction network bandwidth, computer memory, and factory jobs, as well as an
innovative system at Xerox P.A.R.C for trading air-conditioning resources
between buildings. These systems used differing auctions and trader
rational, however, they all relied on a centralised auction mechanism that
required a complete global view of the market. 

%Microeconomic?
%UDC?


\subsection{Summary}

A common factor to all the systems discussed in this section, with the notable
exceptions of Enterprise and Spawn, is that they are centralised in nature,
requiring a auctioneer process.  This centralisation loses some of the
potential benefits of an economic system and imposes severe limitations on
scalability, which is key for any grid-sized market.  

Additionally, the type of auction mechanism used is more often than not some
variation of a call auction which requires a global market view and thus cannot
be decentralised, as well as being highly computationally intensive.

Many are also specifically designed for particular applications and specific
markets, and would be difficult to generalise to a grid type market.

Whilst the basic inspiration for these systems has been economics, they do not
take full advantage of the potential for decentralised markets. This is
understandable, as such markets face implementation challenges, with a central
approach being much simpler initially. They all operate with in  a single
administrative domain, unlike the grid environment, with has multiple
competing domains participating. 

Of the two that utilise a decentralised market (Enterprise and Spawn), both
rely on a local all-to-all communication, which whilst enabling the
decentralisation, is not scalable, especially to non-local resources.
Additionally, both approaches are based on pre-grid concepts of resources and do
not utilise the recent technologies that have been developed by research on
the grid.


\section{Existing Grid Economic Models}
\label{SEC:ECON:GRIDECON}

The idea of an economic model for grid computing is not new. A review of
distributed economic systems is presented in\cite{grid-buyya02-economic} Buyya
\textit{et al} outline the various potential auction strategies for a grid
market place, similar to those outlined here, and provides an overview of some
distributed economic allocation systems. Most of these systems are designed for
a specific purpose, and require administrative control over the resources, and
as such are not applicable to a grid system. 

Nimrod/G \cite{grid-buyya00-nimrod-g}\cite{grid-abramson02-nimrod-g} is the only
economic system designed specifically for the grid environment.  It allows
users to define a maximum cost and execution deadline, and resources define a
base cost. The scheduler uses MDS to discover potential resources and their
associated costs, and uses a cost-deadline optimisation
algorithm\cite{grid-buyya02-dbc} to select suitable resources.  The cost and
estimated completion time for these resource time is passed back to the user,
who can then adjust their cost or deadline amounts if desired, or accept the
allocation and dispatch the job. The resource prices are set manually by
resource owners based upon their own perception of market demand. It does
support advanced reservation in this way.

Nimrod/G thus acts as an economic facilitator with humans as the traders, and
effectively a posted price/retail auction mechanism. This has the potential to
scale well, as retail auctions do, but will be less effective in more dynamic
markets, where faster adaptation to changing supply/demand is required.
Additionally, the trader strategy relies on human interaction, and therefore
is not automatic, and is not as performant or as scalable as
utilising computational agents to automate the bidding process. It is also
based around a centralised resource discovery mechanism (Globus' MDS), which
further limits its scalability.


%%%%%%%%%%%%%%%%%%%%%%%%%%%%%%%%%%%%%%%%%%%%%%%%%%%%%%%%%%%%%%%%%%%%%%%

\section{Summary}

The unique challenges of grid systems present a close match with the challenges
economic systems have evolved to solve, namely a scalable method of allocating
scarce resource amongst self-interested parties. It allows for non-trusting
parties to co-operate in a decentralised and robust manner. This suggests the
economic model of allocation as a good fit to the grid allocation problem.
